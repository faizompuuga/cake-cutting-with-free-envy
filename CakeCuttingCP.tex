\documentclass[12pt]{report}

\title{\textbf{CAKE CUTTING WITH LOW ENVY.}}

\author{}
\renewcommand\thesection{\arabic{section}}
\begin{document}
\maketitle

	\section{Introduction.} 
	\paragraph{Cake cutting with low envy is a fair cutting procedure / algorithm that is used to share something divisible into portions that are satisfying to all partners that what they have got is similar and as good as the other. }
	
		\section{Objective.}
		\paragraph{Before the invention of the cake cutting algorithm, it was unknown whether a finite number of people could share a particular resource in an envy-free manner. The main objective of this algorithm is to create and generate sharing procedures with the lowest extent of envy for large numbers of people. All these procedures are to be tested and evaluated basing on their extent of envy.  }
		
	\section{Background.}
	\paragraph{The stronger criterion of envy-freeness was introduced into the cake-cutting problem by George Gamow and Marvin Stern in 1950s.\\
		A procedure for 3 partners and general pieces was found in 1960. A procedure for 3 partners and connected pieces was found only in 1980.\\
		But for the Envy-Free Division among 4 or more people has been an open problem until 1990s}
	\paragraph{Two lower bounds on the run-time complexity of envy-freeness have been published in the 2000s.}
	\begin{itemize}
		\item For general pieces, the lower bound is $\Omega$(n$^{2}$).
		\item For connected pieces the lower bound is infinity - there is provably no finite protocol for 3 or more partners.
	\end{itemize}
	\paragraph{In the 2010s, several approximation procedures and procedures for special cases have been published. The question whether bounded-time procedures exist for the case of general pieces had remained open for a long time. The problem was finally solved in 2016. Haris Aziz and Simon Mackenzie presented a discrete envy-free protocol that requires \\
		at most ${\displaystyle n^{n^{n^{n^{n^{n}}}}}}$ queries. There is still a very large gap between the lower bound and the procedure. As of August 2016, the exact run-time complexity of envy-freeness is still unknown. }
	\paragraph{For the case of connected pieces, it was noted that the hardness result assumes that the entire cake must be divided. If this requirement is replaced by the weaker requirement that every partner receives a proportional value (at least $\frac{1}{n}$ of the total cake value according to their own valuation), then a bounded procedure for 3 partners is known, but it has remained an open problem whether there exist bounded-time procedures for 4 or more partners.
	}
	\section{Problem statement.}
	\paragraph{Common human behaviour of taking the lion's share during sharing different resources like food, money, land etc. has caused lot of harm and hatred between people that has even lead to unknown deaths. This is all due to lack of a procedure that can be taken to carry out a fair process of the sharing activity. }

	\section{Methodology.}
	\paragraph{}
	\section{Conclusion.}
	\section{References:}
	\begin{itemize}
		\item {How to write a problem statement (2009, march 18th). Retrieved from   http://www.ceptara.com/blog/how-to-write-problem-statement}
		
		\item {https://en.wikipedia.org/wiki/Efficient-cake-cutting}
	\end{itemize}
	
	
	
\end{document}
